\chapter{Resumo Executivo}
\label{cap:ResumoExecutivo}
	
	%\section{Objetivo do Plano de Negócio}
		%A abertura do negócio de entregas de mercadorias possui o objetivo de g
		%Gerar lucros sobre a oportunidades existentes na região onde %o serviço a ser prestado garante suprir a necessidade de %possíveis usuários com problema de locomoção em suas %mercadorias.
		
		%O Plano de negócio é a 
		%Via de informações da empresa para o indivíduo %requerido\footnote{Stakeholders, Empréstimos, etc.}.
		
	\section{A Empresa}
		A empresa, localizada no município de Formiga-MG, tem como objetivo a implantação do serviço on-line \texttt{Entrega aí!}, esse serviço tem como objetivo principal servir de mediador na comunicação entre Transportadores e o Embarcadores de carga. Esses dois últimos são os principais personagens que fundamentam a existência da empresa, respectivamente, pessoas com necessidade de envio de encomendas para outras localidades e pessoas capazes de fornecer um serviço de transporte para essas encomendas.
		
		
		\section{Serviço Ofertado}
		
		A ideia geral do serviço pode ser entendida da seguinte maneira: 
		
		Embarcadores possuem encomendas que desejam enviar para outras localidades, diversos Transportadores, com diversos meios de transporte desejam realizar o frete de encomendas. Partindo então dessa premissa, Transportadores realizam um cadastro no site concordando com os termos e em seguida preenchem seu perfil, ou seja, quais meios de transporte possui à disposição dos Embarcadores, categorias de encomendas que possui interesse ou experiência de transporte deseja enviar. Os Embarcadores por sua vez, realizam o cadastro de forma semelhante ao dos Transportadores, e preenchem um perfil com as informações relativas a esse tipo de usuário. Após o cadastro, Embarcadores passam a ter acesso à plataforma, e assim, podem cadastrar encomendas que necessitam ser transportadas. De acordo com as características das encomendas, essas são exibidas aos Transportadores de maior interesse, que então determinam qual o preço que ele cobrará pelo frete, por último, o Embarcador que cadastrou a encomenda analisa qual o Transportador oferece o melhor serviço, e então o contrata. O serviço cuida de oferecer as melhores sugestões de usuários para seus propósitos (Embarcador para Transportadores e Transportador para Embarcador).
		
		
	\section{O Mercado}
	
		A empresa possui duas visões de mercado, essas foram concebidas para fomentar diferentes fases da empresa, são elas:
	
		\begin{itemize}
			
			\item \textbf{Visão de Mercado Regional:} trata-se da primeira fase da empresa, ou seja, um cenário onde o mercado foi explorado superficialmente, e encontra-se em processo de divulgação e construção da imagem da empresa. Nesse cenário, além de proporcionar o cadastro de veículos de pequeno porte, o sistema suportará o cadastro de veículos de grande porte. Nessa fase, a prioridade será atender as demandas da região onde está localizada sua sede, proporcionando um serviço de qualidade para seus usuários. Agindo dessa forma, é possível oferecer um serviço de qualidade aos usuários, respeitando também as restrições e dificuldades que poderão surgir para uma empresa novata.
			
			
			\item \textbf{Visão de Mercado Nacional:} após um período de consolidação da qualidade dos seus serviços a nível regional, a empresa, deverá almejar novos horizontes, ou seja, passará a atender usuários em todo o território nacional. Será necessário um processo intendo de publicidade fora da área regional da empresa, porém isso não caracteriza um problema, pois, para atingir essa fase a empresa deverá possuir capital social e financeiro a ser investido, admitindo sempre riscos calculados.
			
		\end{itemize}
		
		
	\section{Missão da Empresa, Imagem e Fatores Críticos de Sucesso}
	
		A \textbf{missão} da empresa é definida por \emph{trazer conforto aos Embarcador e produtividade a quem oferta os serviços de frete (Transportador) de forma fácil e prática fornecendo um meio de comunicação eficiente entre eles.}
		
		A \textbf{imagem} da empresa será mantida pela idoneidade e transparência em todos os processo envolvidos. O objetivo é sempre oferecer mecanismos que proporcionem o entendimento de forma clara e verdadeira como são constituídos os processos internos e externos da empresa, criando com isso, uma rede de confiança e consequentemente recomendações.
		
		Os \textbf{fatores críticos} de sucesso da empresa é baseado na constante demanda por serviços de encomenda, um exemplo dessa situação é o crescente mercado de \textit{e-commerce} que indiretamente impacta a rede transportes no Brasil. Como características do serviço, podem ser citados os mecanismos de inovação e auxílio ao Transportador, disponibilizados com o intuito de fornecer uma maior produtividade e menores custos para a todos os usuários.
		
	\section{Estratégia de Venda/Operação}
	
		O dinheiro investido no serviço terá o propósito de auxiliar o crescimento inicial da empresa por algum tempo até que ela se estabilize e possa sustentar suas próprias despesas. Para a venda do serviço, inicialmente, será utilizado um representante no qual realizará a divulgação com as informações necessárias para eliminação de dúvidas de todos os usuários. Este representante realizará a disseminação do serviço para a população pertencente à visão de mercado atual da empresa, a divulgação do serviço também será realizada por meio de mídias sociais a custo zero. Outras formas de publicidade de baixo custo também serão utilizadas, como anúncios na web, panfletos e parceria com postos de gasolina e serviços básicos para a classe atendida.
		
		A medida que os usuários aderem ao serviço, esses divulgarão a seus parceiros de trabalho, fornecendo com isso um serviço indireto e gratuito de publicidade. Na fase nacional do negócio, essa forma de publicidade é maior, pois, transportadores viajam para todo Brasil, tornando a disseminação do serviço facilitada e incisiva.
		
	\section{Elementos de Diferenciação}
	
		O serviço prestado provê ao Embarcador segurança e comparação de fretes anunciados pelos Transportadores. Para os Transportadores oferta serviços de otimização de rota por meio da indicação do melhor caminho a ser percorrido, ou seja, aquele que minimiza as despesas, maximização de lucro por viagem indicando encomendas que podem ser transportadas e que se encontram no trajeto a ser percorrido, e futuramente, indicação de serviços de terceiros (parceiros).
		
	\section{Investimentos}
		
		\subsection{Receitas prévias}
			
			
			% Table generated by Excel2LaTeX from sheet '3-7'
			\begin{table}[htbp]
				\centering
				\caption{Receitas Prévias}
				
				
					\begin{tabularx}{\linewidth}{|X|c|c|}
						\toprule
						\multicolumn{3}{c}{\cellcolor{gray!50}\textbf{Custos Iniciais}} \\
						\midrule
						\textbf{Nome do Serviço} & \textbf{Quantidade} & \textbf{Preço R\$} \\
						\midrule
						\textit{Uniformes}					&	8	& 120,00 \\
						\textit{Materiais de Divulgação}		&	NA	& 400,00 \\
						\textit{Cadastro Empresa}			&	NA	& 400,00 \\
						\textit{Notebook Vendedor}			&	1	& 1.600,00 \\
						\textit{Registro Domínio - 5 Anos}	&	1	& 138,00 \\
						~ & \textbf{Total} & \textbf{2.258,00} \\
						\bottomrule
					\end{tabularx}
				
				
				\label{tab:receitasPrevias2}%
			\end{table}%
			
			
		\subsection{Ponto de equilíbrio (\textit{Break-even})}
			A empresa obterá o equilíbrio nas despesas, o serviço necessita de \emph{20 Caminhões, 34 Carretas, 60 Mistos, 53 Motos e 24 outros}.		
		
		
	\section{Custos Fixos (\textit{Burning rate} - Custos Aprofundados)}
		
		\begin{table}[H]
			\centering
			\caption{Tabela de Custos Fixos.}
			
			
				\begin{tabularx}{\linewidth}{|X|c|c|}
					\toprule
					\multicolumn{3}{c}{\cellcolor{gray!50}\textbf{Custos Fixos}} \\
					\midrule
					\textbf{Nome do Serviço/Produto} & \textbf{Quantidade} & \textbf{Preço em Reais}\\
					\midrule
					\textit{Aluguel Servidor Dedicado} 	& 1 	& 800,00 \\
					\textit{Hospedagem Site - Trienal} 	& 1 	& 10,00 \\
					\textit{Gasolina} 					& 50 lt & 200,00 \\
					\textit{Contador} 					& 1 	& 400,00 \\
					\textit{Vendedor} 					& 1 	& 800,00 \\
					\textit{Desenvolvedores} 			& 3 	& 3.000,00 \\
					\bottomrule
				\end{tabularx}
			
			\label{tab:custosFixos}
		\end{table}