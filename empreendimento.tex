\chapter{O Empreendimento}
\label{cap:OEmpreendimento}
	
	
	\section{Dados da Empresa}
	
		\begin{table}[htbp!]
			\centering
			\caption{Dados da Empresa e Gerentes.}
			\label{fig:processoBusca}
			\includegraphics[width=1\linewidth]{img/1-1.png}
			\fonte{os autores}
		\end{table}
		\FloatBarrier
	
		A empresa, localizada no município de Formiga-MG, tem como objetivo a implantação do serviço on-line Entrega aí! onde realiza a intercomunicação facilitada do transportador e o embarcador. É uma ferramenta de prestação de serviço online onde realizar-se-á o fechamento de contrato entre interessados para que determinada carga seja enviada de um ponto até seu destino com segurança, rapidez e baixo custo.
		
		Na fase inicial da empresa não está prevista uma sede física, essa portanto funcioná de forma \textit{homeoffice} para cada colaborador. Inicialmente está previsto apenas a oferta do serviço online básico, ou seja, fornecer aos usuários um plataforma de qualidade. Entretanto, com a disseminação da tecnologia, popularização dos serviços prestados e boa avaliação da comunidade, será possível incrementar os serviços com meios mais fáceis de acesso, seguros e novidades na prestação de serviços.
        
        Abaixo, na \autoref{fig:logo}, é exibido a logomarca do Entrega Aí! criada pelos dirigentes.
        
        
		\begin{figure}[H]
			\centering
   			\caption{Logomarca da Empresa.}
			\includegraphics[width=0.8\linewidth]{img/logoEntregaAi.png}
			\fonte{os autores}
			\label{fig:logo}
		\end{figure}
		
	\section{Dados dos Dirigentes} \label{sec:dirigentes}
	
		Os dirigentes do negócio são também desenvolvedores do sistema, esses estão a par de como o sistema deve ser construído e utilizado, são eles:
		
			
		\begin{minipage}{0.6\textwidth}
			
			\textbf{João Paulo Fernandes de Cerqueira César}
			
			Bacharel em Ciência de Computação pelo IFMG \textit{campus} Formiga, 22 anos de idade, possui amplo interesse por tecnologia e inovação, coragem e determinação, além de gosto por situações novas e desafiadoras.
				
			
		\end{minipage}
		\begin{minipage}{0.4\textwidth}
			
			\begin{figure}[H]
				\centering
				\includegraphics[width=0.65\textwidth]{img/jp2pb.jpg}
			\end{figure}
			
		\end{minipage}
		
		
		
		\begin{minipage}{0.6\textwidth}
			
			\textbf{Natanael Ramos}
				
			Bacharel em Ciência de Computação pelo IFMG \textit{campus} Formiga, 21 anos de idade, possui conhecimento em desenvolvimento de aplicações \textit{mobile} e amplo interesse em \textit{marketing}, além de ser um empreendedor nato.
				
			
		\end{minipage}
		\begin{minipage}{0.4\textwidth}
			
			\begin{figure}[H]
				\centering
				\includegraphics[width=0.65\textwidth]{img/noelpb.jpg}
			\end{figure}
			
		\end{minipage}
		
		
		
		\begin{minipage}{0.6\textwidth}
			
			\textbf{Rodolfo Labiapari Mansur Guimarães}
			
			Bacharel em Ciência de Computação pelo IFMG \textit{campus} Formiga, 21 anos de idade, possui conhecimento e experiência em desenvolvimento de sistemas web, além de paixão por inclusão digital.
				
			
		\end{minipage}
		\begin{minipage}{0.4\textwidth}
			
			\begin{figure}[H]
				\centering
				\includegraphics[width=0.65\textwidth]{img/dolfpb.jpg}
			\end{figure}
			
		\end{minipage}
				
		%\pagebreak
		
		No início, serão contratados auxiliares para gerenciamento das partes financeiras e contratuais da empresa além de um vendedor que fará representações buscando novos usuários e contratos/convênios, além da contratação de alguma agência de publicidade e propaganda para a disseminação divulgação do serviço de forma mais rápida.
	
	\section{Definição do Negócio}
		A abertura do negócio de entregas de mercadorias possui o objetivo de gerar lucros sobre a oportunidades existentes na região onde o serviço a ser prestado garante suprir a necessidade de possíveis usuários com problema de locomoção em suas mercadorias. O Plano de negócio aqui descrito é uma via de informações da empresa para o indivíduo requerido\footnote{\textit{Stakeholders}, Empréstimos, etc.}.
		
		A ideia geral do serviço pode ser entendida da seguinte maneira: 
		
		Embarcadores possuem encomendas que desejam enviar para outras localidades, diversos Transportadores, com diversos meios de transporte desejam realizar o frete de encomendas. Partindo então dessa premissa, Transportadores realizam um cadastro no site concordando com os termos e em seguida preenchem seu perfil, ou seja, quais meios de transporte possui à disposição dos Embarcadores, categorias de encomendas que possui interesse ou experiência de transporte deseja enviar. Os Embarcadores por sua vez, realizam o cadastro de forma semelhante ao dos Transportadores, e preenchem um perfil com as informações relativas a esse tipo de usuário. Após o cadastro, Embarcadores passam a ter acesso à plataforma, e assim, podem cadastrar encomendas que necessitam ser transportadas. De acordo com as características das encomendas, essas são exibidas aos Transportadores de maior interesse, que então determinam qual o preço que ele cobrará pelo frete, por último, o Embarcador que cadastrou a encomenda analisa qual o Transportador oferece o melhor serviço, e então o contrata. O serviço cuida de oferecer as melhores sugestões de usuários para seus propósitos (Embarcador para Transportadores e Transportador para Embarcador).

	\section{Fontes de Receita}
	
		Os custos iniciais da empresa serão fomentados pelos próprios dirigentes da empresa, devido ao fato de que o empreendimento não necessita de um alto investimento inicial. As metas para o primeiro mês da empresa preveem o pagamento das despesas iniciais e mensais, além disso, com o amadurecimento da empresa ao longo dos anos de serviço, será possível reinvestir na empresa, para assim melhorar os serviços providos. Portanto, nenhum empréstimo será necessário para a abertura da empresa.

	\section{Cenário Futuro para o Mercado}
	
		O fato da locomoção ser um problema em territórios extensos como o território brasileiro, condições de rodovias precárias e funcionamento inconsistente dos Correios, contribuem para o sucesso nacional deste sistema, que terá grande oportunidade de crescimento pela necessidade regional e consequentemente nacional.
		
		Após um período de consolidação da qualidade dos seus serviços a nível regional, a empresa, deverá almejar novos horizontes, ou seja, passará a atender usuários em todo o território nacional. Será necessário um processo intendo de publicidade fora da área regional da empresa, porém isso não caracteriza um problema, pois, para atingir essa fase a empresa deverá possuir capital social e financeiro a ser investido, admitindo sempre riscos calculados.

	\section{Visão}
	
		\begin{itemize}
			
			\item Abrangência geográfica , social e diversificada em relação aos tipos de produtos;
			
			\item Buscar ser referência em um modelo de entregas de qualidade.
			
		\end{itemize}

	\section{Missão}
	
		\begin{itemize}
			
			\item Trazer conforto ao Embarcador e produtividade a
			quem oferta os serviços de frete (Transportador)
			de forma fácil e prática;
			
			\item A empresa fornece um meio de comunicação
			direto entre o Embarcador e o Transportador.
			
		\end{itemize}

	\section{Estratégia}
	
	\subsection{Oportunidades}
	
		\begin{itemize}
			
			\item Disponibilidade de mão de obra técnica (manutenção do	\textit{software});
			
			\item Baixa concorrência na região com mesmo nível de serviços disponíveis;
			
			\item Popularização da internet e ferramentas computacionais;
			
			\item Popularização de ferramentas (web, mapas) para desenvolvimento;
			
			\item Crescimento do \textit{e-commerce};
			
			\item Frequente paralisações dos Correios.
			
		\end{itemize}
		
	\subsection{Ameaças}
	
		\begin{itemize}
			
			\item Restrições de horário dos Transportadores;
			
			\item Confiabilidade no serviço;
			
			\item Serviço sujeito à ineficiência devido a paralisação e qualidade das vias de transporte;
			
			\item Falta de compromisso e responsabilidade de Embarcadores e o Transportador;
			
			\item Dependência do funcionamento de serviços de camada inferior (internet).
			
		\end{itemize}
	
	\subsection{Forças}
		\begin{itemize}
			\item Praticidade de Uso;
			\item Interoperabilidade (compatibilidade com sites de compra);
			\item Privacidade;
			\item Histórico de Serviços (\textit{logs}, preços comparados, etc.);
			\item Categorização de serviços;
			\item Espaço para propaganda de serviços relacionados (Google AdWords, etc.);
			\item Diversificação dos Transportes;
			\item Otimização de roteamento (Caminho mais curto, rápido ou barato);
			\item Rastreamento (ativo ou passivo).
		\end{itemize}
		
		
	\subsection{Fraquezas}
		\begin{itemize}
			\item Comunicação (inclusão digital);
			\item Lucro da empresa por prover serviço pequeno;
			\item Disseminação (Disponibilidade de provedores de serviços, procura de clientes (Embarcadores/Transportadores));
		\end{itemize}
		
	
	\section{Infraestrutura}
		
		O empreendimento não necessita de nenhuma estrutura física para fornecer os serviços devido ao fato de ser \textit{home-office}. A única infraestrutura necessária será a locação de servidores para hospedagem do site e disponibilização dos serviços suprindo a quantidade de acesso necessária a visão de mercado atual da empresa.
	
	\section{Cronograma de Atividades}
	
	% Table generated by Excel2LaTeX from sheet '3-7'
	\begin{table}[H]
		\centering
		\caption{Cronograma de atividades}
		\begin{tabularx}{\linewidth}{|*4{>{\centering\arraybackslash}X|}}
			\toprule
			\textbf{Desenvolvimento do Serviço} & \textbf{Implantação do Serviço de Avaliação de usuário.} & \textbf{Implantação do Serviço de Recomendação de Pacotes.} & \textbf{Implantação do Serviço de Otimização de Rotas.} \\
			\midrule
			\textit{\textbf{Mês 1}} & \textbf{X} & \textbf{} & \textbf{} \\ [1.2ex] \hline
			\textit{\textbf{Mês 2}} & \textbf{X} & \textbf{} & \textbf{} \\ [1.2ex] \hline
			\textit{\textbf{Mês 3}} & \textbf{X} & \textbf{} & \textbf{} \\ [1.2ex] \hline
			\textit{\textbf{Mês 4}} & \textbf{X} & \textbf{X} & \textbf{} \\ [1.2ex] \hline
			\textit{\textbf{Mês 5}} & \textbf{} & \textbf{X} & \textbf{} \\ [1.2ex] \hline
			\textit{\textbf{Mês 6}} & \textbf{} & \textbf{X} & \textbf{X} \\ [1.2ex] \hline
			\textit{\textbf{Mês 7}} & \textbf{} & \textbf{X} & \textbf{X} \\ [1.2ex] \hline
			\textit{\textbf{Mês 8}} & \textbf{} & \textbf{X} & \textbf{X} \\ [1.2ex] \hline
			\textit{\textbf{Mês 9}} & \textbf{} & \textbf{} & \textbf{X} \\ [1.2ex] \hline
			\textit{\textbf{Mês 10}} & \textbf{} & \textbf{} & \textbf{X} \\ [1.2ex] \hline
			\textit{\textbf{Mês 11}} & \textbf{} & \textbf{} & \textbf{X} \\ [1.2ex] \hline
			\textit{\textbf{Mês 12}} & \textbf{} & \textbf{} & \textbf{X} \\ [1.2ex]
			\bottomrule
		\end{tabularx}%
		\label{tab:addlabel}%
	\end{table}%