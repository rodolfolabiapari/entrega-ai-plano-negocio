\chapter{A Proposta de Valor}
	\label{cap:APropostadeValor}
	
	\section{Quadro resumo dos serviços} \label{sec:servicos}
		
        Os preços adotados para os serviços ofertados variam de acordo com a capacidade de cada meio de transporte cadastrado pelo Transportador, e são cobrados mensalmente, como pode ser visto na \autoref{tab:precos}
        
    \begin{table}[H]
		\centering
		\caption{Tabela de preços dos serviços por capacidade.}
          \begin{tabularx}{\linewidth}{|X|c|c|}
          	  \toprule
			  \multicolumn{3}{c}{\cellcolor{gray!50}\textbf{Tabela de Preços por Serviço}} \\
			  \midrule
              \textbf{Veículo}             & \textbf{Preço} & \textbf{Acréscimo}                 \\
              \midrule
              \textit{Carretas}            & $R\$\ 80,00 $ & $+\ R\$\ 20,00 $ por carreta adicional \\ 
              \textit{Caminhões}           & $R\$\ 50,00 $ & ~                        \\
              \textit{Misto (Utilitários)} & $R\$\ 30,00 $ & ~                        \\
              \textit{Motos}               & $R\$\ 10,00 $ & ~                        \\
              \textit{Outros}              & $R\$\ 10,00 $ & ~                        \\
              \bottomrule
          \end{tabularx}
         
		\label{tab:precos}
	\end{table}
	%\FloatBarrier
	
	\section{Caracterização dos serviços}

		O Embarcador se cadastra gratuitamente no nosso serviço online e cadastra um pacote que ele deseja enviar para determinado destino.
		
		O Transportador se cadastra e cadastra pelo menos 1 (um) veículo. Cada veículo possui uma mensalidade de acordo com a tabela citada na Seção \ref{sec:servicos}.
		
		Após os cadastramentos, cada usuário poderá procurar por serviços ou mesmo ser sugeridos serviços por meio do sistema automático. Assim, cada transportador fará sua oferta de frete e o usuário escolhe a que mais lhe interessar.
	
	\section{Diferenciais dos serviços}
    
		\begin{itemize}
        
        	\item Um serviço de rota ótima (mínima, menos pedágios, melhores estradas) será oferecido ao Transportador para que ele obtenha um lucro máximo.
        
			\item Será possível que o embarcador escolha entre várias propostas de frete a melhor que satisfizer suas necessidades e requisições;
			
			\item O transportador poderá usufruir de um serviço onde exibirá outras encomendas que pertencem aquele trajeto que ele percorrerá aumentando seu lucro sobre pacotes transportados.
            
		\end{itemize}
	
	\section{Alianças estratégicas}
    
		O serviço Entrega Aí! usará serviços terceirizados para realizar convênios e propagandas para divulgação e parceria. Por exemplo, pode-se usar de contratos com vários postos de combustíveis, mecânicos, borracheiros, restaurantes, hotéis para que estes forneçam serviços especiais para usuários do Entrega Aí! além destes serem um meio de divulgação do sistema de entregas.
		
		Tais sociedades também podem realizar propaganda nos serviços online do Entrega Aí com a visão de aumentar a venda de seus serviços ou produtos.